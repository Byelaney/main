\documentclass{article}
\usepackage{fullpage}
\begin{document}

\Large{\bf Logistic Regression Notes}  -- Aron Culotta
\normalsize

\section*{From Regression to Classification}
Recall our model of regression with multiple variables:
$$
h(x_i) =  \sum_{j=1}^{k} x_{ij}\theta_j = \theta \cdot  x_i
$$ where $x_{ij}$ is the value of feature $j$ for instance $i$ (e.g.,
``Julie has blonde hair'') and $\theta$ are our model parameters. (The term $\theta \cdot x_i$ is the dot product, $\sum_j \theta_j * x_{ij}$.)

The RSS is:
$$
RSS(h, D) = \frac{1}{2}\sum_{i=1}^{|D|}(y_i - \theta \cdot x_i)^2
$$

The gradient we have to compute is with respect to one of the $\theta$ variables:

$$
\frac{\partial RSS(h, D)}{\partial \theta_j} = \sum_{i=1}^{|D|}(y_i - \theta \cdot x_i)(-x_{ij})
$$

The above assumes that the output variable $y_i$ is a real number. Thus,
this is a model of {\bf regression}. When $y$ is disrete, the
problem is one of {\bf classification}. One could use the linear model
above to do classification. Assume the binary case where $y_i$ can
either be $-1$ or $1$. We can convert the regression model to a
classifier by assuming that if the model outputs a number greater than
0, then predict 1; otherwise, predict -1. However, this can cause
weird things to happen in our update rule:
$$
\theta_j^{t+1} = \theta_j^{t} + \sum_{i=1}^{|D|}(y_i - \theta \cdot x_i)x_{ij}
$$

Suppose the true value of $y_1$ is $1$, and the model (dot product)
returns 2. The instance is technically classified correctly, but the update still counts this as an error of size 1 (since $y_i -\theta \cdot x_i$ is 1). In fact, the update considers this error equivalent to a dot product of $0$, which would in fact result in a classification error.

This is clearly not what we want -- in fact, our gradient descent algorithm may never converge, since we can always make our correct classifications ``more correct'' by cranking up the value of $\theta$.

The way around this is to change our model. Rather than regression, we need classification. We can do this by passing the dot product $x_i \cdot \theta$ through a ``squashing function'' (the logistic function) that ensures its value is always between 0 and 1:
$$
h(x_i) = \frac{1}{1 + e^{-x_i \cdot \theta}}
$$

Because $h(x_i)$ will always be between 0 and 1, we have the right to
call this a {\bf probability} $p(y_i|x_i)$.  This is the conditional
probability of a label $y_i$ given input $x_i$. For the binary case,
assume $y_i$ can be either -1 or 1. Then, we can write
$$
p(y_i=1|x) = \frac{1}{1 + e^{-x_i \cdot \theta}}
$$
for the positive case and
$$
p(y_i=-1|x) = 1 - p(y_i=1|x) = \hbox{ by algebra} = \frac{1}{1 + e^{x_i \cdot \theta}}
$$
for the negative case (note the negative sign is missing before $x_i$ in the negative case).

If we assume $p(y_i|x_i)$ is the probability of {\bf the true label}
for instance $x_i$, we can write this as:
$$
p(y_i|x_i) =  \frac{1}{1 + e^{-y_ix_i \cdot \theta}}
$$ This is the probability of $x_i$ being labeled {\sl correctly} by
the classifier. Now, we can rephrase our learning objective as
maximizing the {\sl joint probability of the true labels for all
  training instances.} Since we assume each instance is drawn
independently, we can write this joint probability as a product of
individual probabilities:
$$
p(y_1 \ldots y_n|x_1 \ldots x_n) = p(y_1|x_n) * p(y_2|x_2) * \ldots * p(y_n|x_n) = \prod_{i=1}^{n}p(y_i|x_i)
$$

Because we're used to minimizing functions using gradient descent, rather than maximizing the probability, we can instead minimize the negative probability. This is our new error function:
$$
E(D, h) = - \prod_{i=1}^{n}p(y_i|x_i)
$$
Note that this is very similar to RSS, but by using probabilities, we ensure that the output for each instance is always between 0 and 1.

Following our learning recipe, our next step is to minimize $E(D,h)$
using gradient descent. Computing the gradient of $E(D,h)$ in its current form is rather hard. So, we can simply transform it to something that's easier to take the gradient of:
$$
E(D,h) = - \ln \prod_{i=1}^n  p(y_i|x_i)
$$ where $\ln$ is the natural log. This equation is called the {\bf
  negative log likelihood}. It turns out that minimizing $f(x)$ or
$\ln f(x)$ results in the same answer, so we can make this
transformation without affecting our final solution.

Now we're ready to calculate the gradient with respect to one parameter $\theta_j$ (close your eyes if necessary):

\begin{eqnarray*}
\frac{\partial E(D,h)}{\partial \theta_j} & = & \frac{\partial}{\partial \theta_j}- \ln \prod_i \frac{1}{1 + e^{-y_i x_i \cdot \theta}} \\
& = &  \frac{\partial}{\partial \theta_j}-  \sum_i \ln \frac{1}{1 + e^{-y_i x_i \cdot \theta}} \hbox{ \hspace{.1in} (by definition of log of products)}\\
& = &  -  \sum_i 1 + e^{-y_i x_i \cdot \theta} \frac{\partial}{\partial \theta_j} \frac{1}{1 + e^{-y_i x_i \cdot \theta}}  \hbox{ \hspace{.1in} (by }\frac{d}{dx}\ln(f(x)) = \frac{1}{f(x)} \frac{d}{dx}f(x) ) \\
& = &  -  \sum_i (1 + e^{-y_i x_i \cdot \theta})\Big(\frac{-y_ix_{ij} e^{-y_ix_i \cdot \theta}}{(1 + e^{-y_ix_i\cdot \theta})^2}\Big) \hbox{ \hspace{.1in} (by quotient and chain rules) }\\
& = & - \sum_i \frac{-y_i x_{ij} e^{-y_i x_i \cdot \theta}}{1 + e^{-y_i x_i \cdot \theta}} \hbox{ \hspace{.1in} (by algebra) }\\
& = & \sum_i y_i x_{ij} (1 - p(y_i | x_i)) \hbox{ \hspace{.1in} } \Big( \hbox{by }\frac{e^{-y_i x_i \cdot \theta}}{1 + e^{-y_i x_i \cdot \theta}} = 1 - p(y_i|x_i) \Big)
\end{eqnarray*}

Thus, the final logistic regression update is:
$$
\theta_j^{t+1} \leftarrow \theta_j^{t} + \eta \sum_i y_i x_{ij}(1-p(y_i|x_i))
$$

Compare this with the update on the previous page, and you can see that we have solved the problem of the unbounded update.

\subsection*{Multiclass Logistic Regression}

Here, we'll extend the above for the multi-class case. Rather than a single parameter vector $\theta$, we will have $m$ vectors, one per class: $\theta = \{\theta^{(1)} \ldots \theta^{(m)}\}$. The probability of a particular class then becomes:
$$
p(y_i=r|x_i) = \frac{e^{x_i \cdot \theta^{(r)}}}{\sum_{k=1}^m e^{x_i \cdot \theta^{(k)}} }
$$

To be more clear, we'll let $y_i^*$ be the true label for instance $x_i$. The error function is then:

$$
E(D, h) = - \prod_{i=1}^{n}p(y_i=y^*|x_i)
$$

\begin{eqnarray*}
\frac{\partial E(D,h)}{\partial \theta_j^{(r)}} & = & \frac{\partial}{\partial \theta_j^{(r)}}- \ln \prod_i \frac{e^{x_i \cdot \theta^{(y^*)}}}{\sum_{k=1}^m e^{x_i \cdot \theta^{(k)}}}\\
& = & \frac{\partial}{\partial \theta_j^{(r)}}- \sum_i \ln \frac{e^{x_i \cdot \theta^{(y^*)}}}{\sum_{k=1}^m e^{x_i \cdot \theta^{(k)}}}\\
\end{eqnarray*}

The numerator in the previous equation is a constant if $y^* \ne r$. So, we'll
compute two derivatives. First, if $y^* = r$:

\begin{eqnarray*}
\frac{\partial E(D,h)}{\partial \theta_j^{(r)}} & = & \sum_i \frac{1}{p(y_i=y^*|x_i)} \frac{\partial}{\partial \theta_j^{(r)}} p(y_i=y^*|x_i)\\
& = & \sum_i \frac{1}{p(y_i=y^*|x_i)} \frac{x_{ij} e^{x_i \theta^{(y^*)}}\sum_{k=1}^m e^{x_i \cdot \theta^{(k)}} - x_{ij}e^{x_i \cdot \theta^{(y^*)}} e^{x_i \cdot \theta^{(r)}}}{(\sum_{k=1}^m e^{x_i \cdot \theta^{(k)}})^2}  \hbox{ \hspace{.1in} (by quotient and chain rules) }\\
& = & \sum_i  \frac{p(y_i=y^*|x_i)x_{ij} (\sum_{k=1}^m  e^{x_i \cdot \theta^{(k)}} -  e^{x_i \cdot \theta^{(r)}})}{p(y_i=y^*|x_i) \sum_{k=1}^m  e^{x_i \cdot \theta^{(k)}}} \hbox{ \hspace{.1in} (by algebra) }\\
& = & \sum_i x_{ij}(1 - p(y_i=r|x_i)) \\
& = & \sum_i x_{ij} -  x_{ij}p(y_i=r|x_i)
\end{eqnarray*}

And we do similar steps when $y^* \ne r$:
\begin{eqnarray*}
\frac{\partial E(D,h)}{\partial \theta_j^{(r)}} & = & \sum_i \frac{1}{p(y_i=y^*|x_i)} \frac{\partial}{\partial \theta_j^{(r)}} p(y_i=y^*|x_i)\\
& = & \sum_i \frac{1}{p(y_i=y^*|x_i)} \frac{-x_{ij}e^{x_i \cdot \theta^{(r)}}  e^{x_i \cdot \theta^{y^*}}}{(\sum_{k=1}^m e^{x_i \cdot \theta^{(k)}})^2} \hbox{ \hspace{.1in} (by reciprocal and chain rules) }\\
& = & \sum_i \frac{p(y_i=y^*|x_i)}{p(y_i=y^*|x_i)} \frac{-x_{ij}e^{x_i \cdot \theta^{(r)}}}{\sum_{k=1}^m e^{x_i \cdot \theta^{(k)}}}  \hbox{ \hspace{.1in} (by algebra) }\\
& = & \sum_i -x_{ij}p(y_i=r|x_i)
\end{eqnarray*}
\end{document}
